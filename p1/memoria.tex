\documentclass[11pt]{article}

%\usepackage{palatino}

\usepackage[utf8]{inputenc}
\usepackage[T1]{fontenc}
% Chivo como en las diapositivas o Fira Sans?
%\usepackage[familydefault,regular]{Chivo}
\usepackage[sfdefault,scaled=.85]{FiraSans}
\usepackage{newtxsf}
\usepackage[spanish]{babel}
\setlength{\parindent}{0pt}
\usepackage{ amssymb }
\usepackage{amsmath}
\usepackage{ wasysym }
\usepackage{color}
\usepackage{graphics}
\usepackage{caption}
\usepackage{lipsum}

\definecolor{text}{RGB}{78,78,78}
\definecolor{accent}{RGB}{129, 26, 24}


%%% PGFPLOTSTABLE

\usepackage{pgfplotstable}


\pgfplotstableset{
columns/0/.style={
     column name={Elementos},
   },
columns/1/.style={
     column name={Tiempo en segundos},
   },
}

\title{Algorítmica: práctica 1 \\ \large Análisis de la eficiencia de algoritmos}
\author{Sofía Almeida Bruno \and Antonio Coín Castro \and María Victoria Granados Pozo \and Miguel Lentisco Ballesteros \and José María Martín Luque}
\date{\today}

\begin{document}
\maketitle

\newpage

\section*{Introducción}
El objetivo de esta práctica es estudiar la eficiencia de los algoritmos que se nos proponen. Para ello realizaremos un estudio teórico y empírico de cada uno. 
El estudio teórico consiste en expresar el número de operaciones \textit{T(n)} requeridas para un problema concreto en función del tamaño \textit{n}. Para el estudio empírico hemos medido los tiempos de ejecución de cada algoritmo para cada uno de los tamaños de las entradas.

\section*{Algoritmos de ordenación}
\subsection*{Burbuja}
Revisa cada elemento de la lista con el siguiente, intercambiándose de posición si no están en el orden correcto.
\subsection*{Insercción}
Consideramos el elemento N-ésimo de la lista y lo ordenamos respecto de los elementos desde el primero hasta el N-1-ésimo.

\subsection*{Selección}
Consiste en encontrar el menor de todos los elementos de la lista e intercambiarlo con el de la primera posición. Luego con el segundo, y así sucesivamente hasta ordenarlo todo.

\subsection*{Mergesort}
Se basa en la técnica de divide y vencerás. Consiste en dividir la lista en sublistas de la mitad de tamaño, ordenando cada una de ellas de forma recursiva. Si el tamaño de la lista es 0 o 1 la lista ya está ordenada. Para acabar juntamos todas las sublistas en una sola.


\subsection*{Quicksort}
También se basa en la técnica de divide y vencerás.
En primer lugar elegimos un elemento de la lista, que llamaremos \textit{pivote}. A continución los elementos de la lista se ordenarán de forma que la derecha del pivote queden los mayores y a la izquierda los menores. De esta forma dividimos la lista en dos sublistas, la de la derecha y la de la izquierda. Repetiremos el proceso mientras las sublistas tengan más de un elemento.

\subsection*{Heapsort}
Método de ordenación por selección.
\textit{Heap} es un árbol binario de altura mínima, en el que los nodos del nivel más bajo están lo más a la izquierda posible. Además los hijos de cada nodo son siempre menores que el padre. De está forma no hay que recorrer el árbol de forma desordenada para encontrar los elementos máximos.

\section*{Otros algoritmos}

\subsection*{Floyd}
Algoritmo de análisis sobre grados para encontrar el camino mínimo en grafos ponderados.

\subsection*{Hanoi}
El número de pasos crece exponencialmente con el número de discos. Hay tres pilas de númeor origen (ordenada), auxiliar y destino. Se moverá un disco de la pila origen a la destino si hay un disco en la pila origen. En caso contrario se moverán los discos a la auxiliar menos la más grnade.
Por último moveremos el disco mayor al destino y mover las n-1 encima de la más grande.

\section*{Cálculo de la eficiencia empírica}


\begin{center}
	\input{graficos/burbuja-linux-O0}
\end{center}



\begin{center}
	\input{graficos/floyd-linux-O0}
\end{center}



\begin{center}
	\input{graficos/hanoi-linux-O0}
\end{center}



\begin{center}
	\input{graficos/heapsort-linux-O0}
\end{center}



\begin{center}
	\input{graficos/insercion-linux-O0}
\end{center}



\begin{center}
	\input{graficos/mergesort-linux-O0}
\end{center}



\begin{center}
	\input{graficos/quicksort-linux-O0}
\end{center}


\begin{center}
	\input{graficos/seleccion-linux-O0}
\end{center}

\begin{center}
	\input{graficos/ajuste-hanoi}
\end{center}

\newpage

Algunas tablas:


\pgfplotstableread{datos/burbuja_datos/burbuja-linux-O0.dat}\burbujalinuxOCero
\pgfplotstableread{datos/seleccion_datos/seleccion-linux-O0.dat}\seleccionlinuxOCero
\pgfplotstableread{datos/insercion_datos/insercion-linux-O0.dat}\insercionlinuxOCero

\pgfplotstablecreatecol[copy column from table={\burbujalinuxOCero}{[index] 1}] {Burbuja} {\burbujalinuxOCero}
\pgfplotstablecreatecol[copy column from table={\seleccionlinuxOCero}{[index] 1}] {Selección} {\burbujalinuxOCero}
\pgfplotstablecreatecol[copy column from table={\insercionlinuxOCero}{[index] 1}] {Inserción} {\burbujalinuxOCero}

\begin{figure*}[h]
	\centering
	\caption*{Algoritmos que son $O(n^2)$ (tiempos en segundos)}
	\pgfplotstabletypeset[columns={0, Burbuja, Selección, Inserción}]{\burbujalinuxOCero}
\end{figure*}

\pgfplotstableread{datos/mergesort_datos/mergesort-linux-O0.dat}\mergesortlinuxOCero
\pgfplotstableread{datos/quicksort_datos/quicksort-linux-O0.dat}\quicksortlinuxOCero
\pgfplotstableread{datos/heapsort_datos/heapsort-linux-O0.dat}\heapsortlinuxOCero

\pgfplotstablecreatecol[copy column from table={\mergesortlinuxOCero}{[index] 1}] {Mergesort} {\mergesortlinuxOCero}
\pgfplotstablecreatecol[copy column from table={\quicksortlinuxOCero}{[index] 1}] {Quicksort} {\mergesortlinuxOCero}
\pgfplotstablecreatecol[copy column from table={\heapsortlinuxOCero}{[index] 1}] {Heapsort} {\mergesortlinuxOCero}

\begin{figure*}[h]
	\centering
	\caption*{Algoritmos que son $O(nlog(n))$ (tiempos en segundos)}
	\pgfplotstabletypeset[columns={0, Mergesort, Quicksort, Heapsort}]{\mergesortlinuxOCero}
\end{figure*}



\section*{Cálculo de la eficiencia híbrida}

\end{document}

