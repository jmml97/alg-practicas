%%
% Please see https://bitbucket.org/rivanvx/beamer/wiki/Home for obtaining beamer.
%%
\documentclass[spanish]{beamer}

%%% CODIFICACIÓN

\usepackage[utf8]{inputenc}
\usepackage[spanish]{babel}

%%% FUENTES

\usepackage[T1]{fontenc}
\usepackage[familydefault,regular]{Chivo}
\usepackage{newtxsf} % Fuente de matemáticas

\setbeamertemplate{navigation symbols}{}

%%% COLORES

\definecolor{background}{RGB}{237,237,237}
\definecolor{text}{RGB}{78,78,78}
\definecolor{accent}{RGB}{129, 26, 24}


%%% AJUSTES DE BEAMER

% ¿Negrita en el título de diapositiva o no?
%\setbeamertemplate{frametitle}{\color{accent}\vspace*{1cm}\bfseries\insertframetitle\par\vskip-6pt}

\setbeamertemplate{frametitle}{\color{accent}\vspace*{1cm}\insertframetitle\par\vskip-6pt}

\setbeamertemplate{itemize items}[circle] % Viñetas de itemize

%%% CONFIGURACIÓN DE COLORES DE BEAMER

\setbeamercolor{background canvas}{bg=background}
\setbeamercolor{normal text}{fg=text}
\setbeamercolor{alerted text}{fg=accent}
\setbeamercolor{block title}{fg=accent}
\setbeamercolor{alerted text}{fg=accent}
\setbeamercolor{itemize item}{fg=accent}
\setbeamercolor{enumerate item}{fg=accent}
\setbeamercolor*{title}{fg=accent}
\setbeamercolor{qed symbol}{fg=accent}
\usebeamercolor[fg]{normal text}

%%% PGFPLOTSTABLE

\usepackage{pgfplotstable}


\pgfplotstableset{
columns/0/.style={
     column name={Elementos},
   },
columns/1/.style={
     column name={Tiempo en segundos},
   },
}

%%% INFORMACIÓN DEL DOCUMENTO

\title{Algorítmica: práctica 1}
\subtitle{Análisis de la eficiencia de algoritmos}
\author{Sofía Almeida Bruno\\ Antonio Coín Castro\\ María Victoria Granados Pozo\\ Miguel Lentisco Ballesteros\\ José María Martín Luque}
\begin{document}




\maketitle

\begin{frame}{Objetivo}
	Estudiar tanto la eficiencia teórica como la eficiencia empírica de 8 algoritmos.
	
	\begin{itemize}
		\item Burbuja, Inserción y Selección
		\item Mergesort, Quicksort y Heapsort
		\item Floyd y Hanoi
	\end{itemize}
\end{frame}

\begin{frame}{Burbuja}
	Revisa cada elemento de la lista con el siguiente, intercambiándose de posición si no están en el orden correcto.
	
	\vskip 0.5cm
	
	Es $O(n^2)$.
	
\end{frame}

\begin{frame}{}
	\begin{center}
		\input{graficos/burbuja-linux-O0}
	\end{center}
\end{frame}

\begin{frame}{Inserción}
	Consideramos el elemento N-ésimo de la lista y lo ordenamos respecto de los elementos desde el primero hasta el N-1-ésimo.
	
	\vskip 0.5cm
	
	Es $O(n^2)$
\end{frame}

\begin{frame}
	\begin{center}
		\input{graficos/insercion-linux-O0}
	\end{center}
\end{frame}

\begin{frame}{Selección}
	Consiste en encontrar el menor de todos los elementos de la lista e intercambiarlo con el de la primera posición. Luego con el segundo, y así sucesivamente hasta ordenarlo todo.
	\vskip 0.5cm
	
	Es $O(n^2)$
\end{frame}

\begin{frame}
	\begin{center}
		\input{graficos/seleccion-linux-O0}
	\end{center}
\end{frame}

\begin{frame}{Mergesort}

	Se basa en la técnica de \textit{divide y vencerás}.

	\vskip 0.5cm
	
	\begin{itemize}
		\item Se divide la lista a ordenar en dos sublistas de la mitad de tamaño.
		\item Se ordena cada sublista de forma recursiva.
		\item Si el tamaño de una sublista es 0 o 1 entonces ya está ordenada.
		\item Se unen todas las sublistas en una sola.
	\end{itemize}
	
	\vskip 0.5cm
	
	Es $O(nlogn)$
\end{frame}

\begin{frame}
	\begin{center}
		\input{graficos/mergesort-linux-O0}
	\end{center}
\end{frame}

\begin{frame}{Quicksort}

	Se basa en la técnica de \textit{divide y vencerás}.

	\vskip 0.5cm
	
	\begin{itemize}
		\item Elegimos un elemento de la lista, el \textit{pivote}.
		\item Se ordena la lista, dejando los elementos mayores a la derecha del pivote y los menores a la izquierda.
		\item Realizamos el proceso recursivamente en las dos sublistas que nos quedan (derecha e izquierda) hasta que tengan 0 o 1 elemento.
	\end{itemize}
	
	\vskip 0.5cm
	
	Es $O(nlogn)$
\end{frame}

\begin{frame}
	\begin{center}
		\input{graficos/quicksort-linux-O0}
	\end{center}
\end{frame}

\begin{frame}{Heapsort}

	Es un método de ordenación por selección.

	\vskip 0.5cm
	
	\begin{itemize}
		\item El \textit{heap} es un árbol binario de altura mínima, en el que los nodos del nivel más bajo están lo más a la izquierda posible.
		\item Los hijos de cada nodo son siempre menores que el padre.
		\item No es necesario recorrer el árbol de forma desordenada para encontrar los elementos máximos.
	\end{itemize}
	
	\vskip 0.5cm
	
	Es $O(nlogn)$
\end{frame}


\begin{frame}
	\begin{center}
		\input{graficos/heapsort-linux-O0}
	\end{center}
\end{frame}

\begin{frame}{Floyd}

	Algoritmo de análisis sobre grados para encontrar el camino mínimo en grafos ponderados.

	\vskip 0.5cm
	
	\begin{itemize}
		\item ¿Explicación del algoritmo?
	\end{itemize}
	
	\vskip 0.5cm
	
	Es $O(n^3)$
\end{frame}

\begin{frame}
	\begin{center}
		\input{graficos/floyd-linux-O0}
	\end{center}
\end{frame}

\begin{frame}{Hanoi}

	¿Qué es?

	\vskip 0.5cm
	
	\begin{itemize}
		\item ¿Explicación del algoritmo?
	\end{itemize}
	
	\vskip 0.5cm
	
	Es $O(2^n)$
\end{frame}

\begin{frame}
	\begin{center}
		\input{graficos/hanoi-linux-O0}
	\end{center}
\end{frame}

\begin{frame}{Datos de la eficiencia empírica}
	Hemos recopilado los datos de la eficiencia empírica de la ejecución de los distintos algoritmos en varias tablas comparativas.
	
	\vskip 0.5cm
	
	Hemos creado tablas para los distintos órdenes de eficiencia de los algoritmos y hemos puesto juntos aquellos que tienen el mismo.
	
	\vskip 0.5cm
	
	Finalmente, para cada tabla comparativa hemos creado una gráfica.
	
\end{frame}

%%% TABLAS ALGORITMOS O(n2) CON OPTIMIZACIÓN O0

\pgfplotstableread{datos/burbuja_datos/burbuja-linux-O0.dat}\burbujalinuxOCero
	\pgfplotstableread{datos/seleccion_datos/seleccion-linux-O0.dat}\seleccionlinuxOCero
	\pgfplotstableread{datos/insercion_datos/insercion-linux-O0.dat}\insercionlinuxOCero
	
	\pgfplotstablecreatecol[copy column from table={\burbujalinuxOCero}{[index] 1}] {Burbuja} {\burbujalinuxOCero}
	\pgfplotstablecreatecol[copy column from table={\seleccionlinuxOCero}{[index] 1}] {Selección} {\burbujalinuxOCero}
	\pgfplotstablecreatecol[copy column from table={\insercionlinuxOCero}{[index] 1}] {Inserción} {\burbujalinuxOCero}

\begin{frame}
	\begin{columns}[c] % align columns
		\begin{column}{.6\textwidth}
			\resizebox*{!}{9cm}{
				\pgfplotstabletypeset[columns={0, Burbuja, Inserción, Selección}]{\burbujalinuxOCero}
			}
		\end{column}%
		
		\hfill%
		
		\begin{column}{.3\textwidth}
			Algoritmos que son $O(n^2)$
		\end{column}%
		
	\end{columns}
\end{frame}

\begin{frame}
	\begin{center}
		\input{graficos/ncuadrado}
	\end{center}
\end{frame}

%%% TABLAS ALGORITMOS O(nlogn) CON OPTIMIZACIÓN O0

\pgfplotstableread{datos/mergesort_datos/mergesort-linux-O0.dat}\mergesortlinuxOCero
	\pgfplotstableread{datos/quicksort_datos/quicksort-linux-O0.dat}\quicksortlinuxOCero
	\pgfplotstableread{datos/heapsort_datos/heapsort-linux-O0.dat}\heapsortlinuxOCero
	
	\pgfplotstablecreatecol[copy column from table={\mergesortlinuxOCero}{[index] 1}] {Mergesort} {\mergesortlinuxOCero}
	\pgfplotstablecreatecol[copy column from table={\quicksortlinuxOCero}{[index] 1}] {Quicksort} {\mergesortlinuxOCero}
	\pgfplotstablecreatecol[copy column from table={\heapsortlinuxOCero}{[index] 1}] {Heapsort} {\mergesortlinuxOCero}

\begin{frame}
	\begin{columns}[c] % align columns
		\begin{column}{.6\textwidth}
			\resizebox*{!}{9cm}{
				\pgfplotstabletypeset[columns={0, Mergesort, Quicksort, Heapsort}]{\mergesortlinuxOCero}
			}
		\end{column}%
		
		\hfill%
		
		\begin{column}{.3\textwidth}
			Algoritmos que son $O(nlogn)$
		\end{column}%
		
	\end{columns}
\end{frame}

\begin{frame}
	\begin{center}
		\input{graficos/nlogn}
	\end{center}
\end{frame}

%%% TABLA FLOYD
\begin{frame}
	\begin{columns}[c] % align columns
		\begin{column}{.6\textwidth}
			\resizebox*{!}{9cm}{
				\pgfplotstabletypeset[]{./datos/floyd_datos/floyd-linux-O0.dat}
			}
		\end{column}%
		
		\hfill%
		
		\begin{column}{.3\textwidth}
			Floyd
		\end{column}%
		
	\end{columns}
\end{frame}

%%% TABLA HANOI
\begin{frame}
	\begin{columns}[c] % align columns
		\begin{column}{.6\textwidth}
			\resizebox*{!}{9cm}{
				\pgfplotstabletypeset[]{./datos/hanoi_datos/hanoi-linux-O0.dat}
			}
		\end{column}%
		
		\hfill%
		
		\begin{column}{.3\textwidth}
			Hanoi
		\end{column}%
		
	\end{columns}
\end{frame}

\begin{frame}{Comparativa de los algoritmos de ordenación}
	Recopilando los datos de todos los algoritmos de ordenación hemos realizado una tabla comparativa en la que se muestra qué algoritmo es el más eficiente.
\end{frame}

\begin{frame}
	\begin{center}
		% GNUPLOT: LaTeX picture with Postscript
\begingroup
  \makeatletter
  \providecommand\color[2][]{%
    \GenericError{(gnuplot) \space\space\space\@spaces}{%
      Package color not loaded in conjunction with
      terminal option `colourtext'%
    }{See the gnuplot documentation for explanation.%
    }{Either use 'blacktext' in gnuplot or load the package
      color.sty in LaTeX.}%
    \renewcommand\color[2][]{}%
  }%
  \providecommand\includegraphics[2][]{%
    \GenericError{(gnuplot) \space\space\space\@spaces}{%
      Package graphicx or graphics not loaded%
    }{See the gnuplot documentation for explanation.%
    }{The gnuplot epslatex terminal needs graphicx.sty or graphics.sty.}%
    \renewcommand\includegraphics[2][]{}%
  }%
  \providecommand\rotatebox[2]{#2}%
  \@ifundefined{ifGPcolor}{%
    \newif\ifGPcolor
    \GPcolortrue
  }{}%
  \@ifundefined{ifGPblacktext}{%
    \newif\ifGPblacktext
    \GPblacktextfalse
  }{}%
  % define a \g@addto@macro without @ in the name:
  \let\gplgaddtomacro\g@addto@macro
  % define empty templates for all commands taking text:
  \gdef\gplbacktext{}%
  \gdef\gplfronttext{}%
  \makeatother
  \ifGPblacktext
    % no textcolor at all
    \def\colorrgb#1{}%
    \def\colorgray#1{}%
  \else
    % gray or color?
    \ifGPcolor
      \def\colorrgb#1{\color[rgb]{#1}}%
      \def\colorgray#1{\color[gray]{#1}}%
      \expandafter\def\csname LTw\endcsname{\color{white}}%
      \expandafter\def\csname LTb\endcsname{\color{black}}%
      \expandafter\def\csname LTa\endcsname{\color{black}}%
      \expandafter\def\csname LT0\endcsname{\color[rgb]{1,0,0}}%
      \expandafter\def\csname LT1\endcsname{\color[rgb]{0,1,0}}%
      \expandafter\def\csname LT2\endcsname{\color[rgb]{0,0,1}}%
      \expandafter\def\csname LT3\endcsname{\color[rgb]{1,0,1}}%
      \expandafter\def\csname LT4\endcsname{\color[rgb]{0,1,1}}%
      \expandafter\def\csname LT5\endcsname{\color[rgb]{1,1,0}}%
      \expandafter\def\csname LT6\endcsname{\color[rgb]{0,0,0}}%
      \expandafter\def\csname LT7\endcsname{\color[rgb]{1,0.3,0}}%
      \expandafter\def\csname LT8\endcsname{\color[rgb]{0.5,0.5,0.5}}%
    \else
      % gray
      \def\colorrgb#1{\color{black}}%
      \def\colorgray#1{\color[gray]{#1}}%
      \expandafter\def\csname LTw\endcsname{\color{white}}%
      \expandafter\def\csname LTb\endcsname{\color{black}}%
      \expandafter\def\csname LTa\endcsname{\color{black}}%
      \expandafter\def\csname LT0\endcsname{\color{black}}%
      \expandafter\def\csname LT1\endcsname{\color{black}}%
      \expandafter\def\csname LT2\endcsname{\color{black}}%
      \expandafter\def\csname LT3\endcsname{\color{black}}%
      \expandafter\def\csname LT4\endcsname{\color{black}}%
      \expandafter\def\csname LT5\endcsname{\color{black}}%
      \expandafter\def\csname LT6\endcsname{\color{black}}%
      \expandafter\def\csname LT7\endcsname{\color{black}}%
      \expandafter\def\csname LT8\endcsname{\color{black}}%
    \fi
  \fi
    \setlength{\unitlength}{0.0500bp}%
    \ifx\gptboxheight\undefined%
      \newlength{\gptboxheight}%
      \newlength{\gptboxwidth}%
      \newsavebox{\gptboxtext}%
    \fi%
    \setlength{\fboxrule}{0.5pt}%
    \setlength{\fboxsep}{1pt}%
\begin{picture}(5760.00,4320.00)%
    \gplgaddtomacro\gplbacktext{%
      \colorrgb{0.30,0.30,0.30}%
      \put(1650,1246){\makebox(0,0)[r]{\strut{}$\textcolor{text}{0.0001}$}}%
      \colorrgb{0.30,0.30,0.30}%
      \put(1650,1648){\makebox(0,0)[r]{\strut{}$\textcolor{text}{0.001}$}}%
      \colorrgb{0.30,0.30,0.30}%
      \put(1650,2050){\makebox(0,0)[r]{\strut{}$\textcolor{text}{0.01}$}}%
      \colorrgb{0.30,0.30,0.30}%
      \put(1650,2453){\makebox(0,0)[r]{\strut{}$\textcolor{text}{0.1}$}}%
      \colorrgb{0.30,0.30,0.30}%
      \put(1650,2855){\makebox(0,0)[r]{\strut{}$\textcolor{text}{1}$}}%
      \colorrgb{0.30,0.30,0.30}%
      \put(1650,3257){\makebox(0,0)[r]{\strut{}$\textcolor{text}{10}$}}%
      \colorrgb{0.30,0.30,0.30}%
      \put(1650,3659){\makebox(0,0)[r]{\strut{}$\textcolor{text}{100}$}}%
      \colorrgb{0.30,0.30,0.30}%
      \put(1782,1114){\rotatebox{45}{\makebox(0,0)[r]{\strut{}$\textcolor{text}{0}$}}}%
      \colorrgb{0.30,0.30,0.30}%
      \put(2230,1114){\rotatebox{45}{\makebox(0,0)[r]{\strut{}$\textcolor{text}{20000}$}}}%
      \colorrgb{0.30,0.30,0.30}%
      \put(2677,1114){\rotatebox{45}{\makebox(0,0)[r]{\strut{}$\textcolor{text}{40000}$}}}%
      \colorrgb{0.30,0.30,0.30}%
      \put(3125,1114){\rotatebox{45}{\makebox(0,0)[r]{\strut{}$\textcolor{text}{60000}$}}}%
      \colorrgb{0.30,0.30,0.30}%
      \put(3573,1114){\rotatebox{45}{\makebox(0,0)[r]{\strut{}$\textcolor{text}{80000}$}}}%
      \colorrgb{0.30,0.30,0.30}%
      \put(4020,1114){\rotatebox{45}{\makebox(0,0)[r]{\strut{}$\textcolor{text}{100000}$}}}%
      \colorrgb{0.30,0.30,0.30}%
      \put(4468,1114){\rotatebox{45}{\makebox(0,0)[r]{\strut{}$\textcolor{text}{120000}$}}}%
      \colorrgb{0.30,0.30,0.30}%
      \put(4915,1114){\rotatebox{45}{\makebox(0,0)[r]{\strut{}$\textcolor{text}{140000}$}}}%
      \colorrgb{0.30,0.30,0.30}%
      \put(5363,1114){\rotatebox{45}{\makebox(0,0)[r]{\strut{}$\textcolor{text}{160000}$}}}%
    }%
    \gplgaddtomacro\gplfronttext{%
      \colorrgb{0.30,0.30,0.30}%
      \put(220,2452){\rotatebox{-270}{\makebox(0,0){\strut{}Tiempo de ejecución (s)}}}%
      \colorrgb{0.30,0.30,0.30}%
      \put(3572,220){\makebox(0,0){\strut{}Tamaño del vector (elementos)}}%
      \colorrgb{0.30,0.30,0.30}%
      \put(3572,3989){\makebox(0,0){\strut{}Algoritmos de ordenación}}%
      \csname LTb\endcsname%
      \put(4376,3486){\makebox(0,0)[r]{\strut{}Mergesort}}%
      \csname LTb\endcsname%
      \put(4376,3266){\makebox(0,0)[r]{\strut{}Heapsort}}%
      \csname LTb\endcsname%
      \put(4376,3046){\makebox(0,0)[r]{\strut{}Quicksort}}%
      \csname LTb\endcsname%
      \put(4376,2826){\makebox(0,0)[r]{\strut{}Burbuja}}%
      \csname LTb\endcsname%
      \put(4376,2606){\makebox(0,0)[r]{\strut{}Inserción}}%
      \csname LTb\endcsname%
      \put(4376,2386){\makebox(0,0)[r]{\strut{}Selección}}%
    }%
    \gplbacktext
    \put(0,0){\includegraphics{./graficos/ordenacion}}%
    \gplfronttext
  \end{picture}%
\endgroup

	\end{center}
\end{frame}

\begin{frame}
	\fontsize{8pt}{7.2}\selectfont
	\begin{center}
		\input{graficos/ajuste-burbuja}
	\end{center}
\end{frame}

\begin{frame}
	\fontsize{8pt}{7.2}\selectfont
	\begin{center}
		\input{graficos/ajuste-insercion}
	\end{center}
\end{frame}

\begin{frame}
	\fontsize{8pt}{7.2}\selectfont
	\begin{center}
		\input{graficos/ajuste-seleccion}
	\end{center}
\end{frame}

\begin{frame}
	\fontsize{8pt}{7.2}\selectfont
	\begin{center}
		% GNUPLOT: LaTeX picture with Postscript
\begingroup
  \makeatletter
  \providecommand\color[2][]{%
    \GenericError{(gnuplot) \space\space\space\@spaces}{%
      Package color not loaded in conjunction with
      terminal option `colourtext'%
    }{See the gnuplot documentation for explanation.%
    }{Either use 'blacktext' in gnuplot or load the package
      color.sty in LaTeX.}%
    \renewcommand\color[2][]{}%
  }%
  \providecommand\includegraphics[2][]{%
    \GenericError{(gnuplot) \space\space\space\@spaces}{%
      Package graphicx or graphics not loaded%
    }{See the gnuplot documentation for explanation.%
    }{The gnuplot epslatex terminal needs graphicx.sty or graphics.sty.}%
    \renewcommand\includegraphics[2][]{}%
  }%
  \providecommand\rotatebox[2]{#2}%
  \@ifundefined{ifGPcolor}{%
    \newif\ifGPcolor
    \GPcolortrue
  }{}%
  \@ifundefined{ifGPblacktext}{%
    \newif\ifGPblacktext
    \GPblacktextfalse
  }{}%
  % define a \g@addto@macro without @ in the name:
  \let\gplgaddtomacro\g@addto@macro
  % define empty templates for all commands taking text:
  \gdef\gplbacktext{}%
  \gdef\gplfronttext{}%
  \makeatother
  \ifGPblacktext
    % no textcolor at all
    \def\colorrgb#1{}%
    \def\colorgray#1{}%
  \else
    % gray or color?
    \ifGPcolor
      \def\colorrgb#1{\color[rgb]{#1}}%
      \def\colorgray#1{\color[gray]{#1}}%
      \expandafter\def\csname LTw\endcsname{\color{white}}%
      \expandafter\def\csname LTb\endcsname{\color{black}}%
      \expandafter\def\csname LTa\endcsname{\color{black}}%
      \expandafter\def\csname LT0\endcsname{\color[rgb]{1,0,0}}%
      \expandafter\def\csname LT1\endcsname{\color[rgb]{0,1,0}}%
      \expandafter\def\csname LT2\endcsname{\color[rgb]{0,0,1}}%
      \expandafter\def\csname LT3\endcsname{\color[rgb]{1,0,1}}%
      \expandafter\def\csname LT4\endcsname{\color[rgb]{0,1,1}}%
      \expandafter\def\csname LT5\endcsname{\color[rgb]{1,1,0}}%
      \expandafter\def\csname LT6\endcsname{\color[rgb]{0,0,0}}%
      \expandafter\def\csname LT7\endcsname{\color[rgb]{1,0.3,0}}%
      \expandafter\def\csname LT8\endcsname{\color[rgb]{0.5,0.5,0.5}}%
    \else
      % gray
      \def\colorrgb#1{\color{black}}%
      \def\colorgray#1{\color[gray]{#1}}%
      \expandafter\def\csname LTw\endcsname{\color{white}}%
      \expandafter\def\csname LTb\endcsname{\color{black}}%
      \expandafter\def\csname LTa\endcsname{\color{black}}%
      \expandafter\def\csname LT0\endcsname{\color{black}}%
      \expandafter\def\csname LT1\endcsname{\color{black}}%
      \expandafter\def\csname LT2\endcsname{\color{black}}%
      \expandafter\def\csname LT3\endcsname{\color{black}}%
      \expandafter\def\csname LT4\endcsname{\color{black}}%
      \expandafter\def\csname LT5\endcsname{\color{black}}%
      \expandafter\def\csname LT6\endcsname{\color{black}}%
      \expandafter\def\csname LT7\endcsname{\color{black}}%
      \expandafter\def\csname LT8\endcsname{\color{black}}%
    \fi
  \fi
    \setlength{\unitlength}{0.0500bp}%
    \ifx\gptboxheight\undefined%
      \newlength{\gptboxheight}%
      \newlength{\gptboxwidth}%
      \newsavebox{\gptboxtext}%
    \fi%
    \setlength{\fboxrule}{0.5pt}%
    \setlength{\fboxsep}{1pt}%
\begin{picture}(5760.00,4320.00)%
    \gplgaddtomacro\gplbacktext{%
      \colorrgb{0.30,0.30,0.30}%
      \put(1518,1246){\makebox(0,0)[r]{\strut{}$\textcolor{text}{0}$}}%
      \colorrgb{0.30,0.30,0.30}%
      \put(1518,1591){\makebox(0,0)[r]{\strut{}$\textcolor{text}{0.005}$}}%
      \colorrgb{0.30,0.30,0.30}%
      \put(1518,1935){\makebox(0,0)[r]{\strut{}$\textcolor{text}{0.01}$}}%
      \colorrgb{0.30,0.30,0.30}%
      \put(1518,2280){\makebox(0,0)[r]{\strut{}$\textcolor{text}{0.015}$}}%
      \colorrgb{0.30,0.30,0.30}%
      \put(1518,2625){\makebox(0,0)[r]{\strut{}$\textcolor{text}{0.02}$}}%
      \colorrgb{0.30,0.30,0.30}%
      \put(1518,2970){\makebox(0,0)[r]{\strut{}$\textcolor{text}{0.025}$}}%
      \colorrgb{0.30,0.30,0.30}%
      \put(1518,3314){\makebox(0,0)[r]{\strut{}$\textcolor{text}{0.03}$}}%
      \colorrgb{0.30,0.30,0.30}%
      \put(1518,3659){\makebox(0,0)[r]{\strut{}$\textcolor{text}{0.035}$}}%
      \colorrgb{0.30,0.30,0.30}%
      \put(1650,1114){\rotatebox{45}{\makebox(0,0)[r]{\strut{}$\textcolor{text}{0}$}}}%
      \colorrgb{0.30,0.30,0.30}%
      \put(2114,1114){\rotatebox{45}{\makebox(0,0)[r]{\strut{}$\textcolor{text}{20000}$}}}%
      \colorrgb{0.30,0.30,0.30}%
      \put(2578,1114){\rotatebox{45}{\makebox(0,0)[r]{\strut{}$\textcolor{text}{40000}$}}}%
      \colorrgb{0.30,0.30,0.30}%
      \put(3042,1114){\rotatebox{45}{\makebox(0,0)[r]{\strut{}$\textcolor{text}{60000}$}}}%
      \colorrgb{0.30,0.30,0.30}%
      \put(3507,1114){\rotatebox{45}{\makebox(0,0)[r]{\strut{}$\textcolor{text}{80000}$}}}%
      \colorrgb{0.30,0.30,0.30}%
      \put(3971,1114){\rotatebox{45}{\makebox(0,0)[r]{\strut{}$\textcolor{text}{100000}$}}}%
      \colorrgb{0.30,0.30,0.30}%
      \put(4435,1114){\rotatebox{45}{\makebox(0,0)[r]{\strut{}$\textcolor{text}{120000}$}}}%
      \colorrgb{0.30,0.30,0.30}%
      \put(4899,1114){\rotatebox{45}{\makebox(0,0)[r]{\strut{}$\textcolor{text}{140000}$}}}%
      \colorrgb{0.30,0.30,0.30}%
      \put(5363,1114){\rotatebox{45}{\makebox(0,0)[r]{\strut{}$\textcolor{text}{160000}$}}}%
    }%
    \gplgaddtomacro\gplfronttext{%
      \colorrgb{0.30,0.30,0.30}%
      \put(220,2452){\rotatebox{-270}{\makebox(0,0){\strut{}Tiempo de ejecución (s)}}}%
      \colorrgb{0.30,0.30,0.30}%
      \put(3506,220){\makebox(0,0){\strut{}Tamaño del vector (elementos)}}%
      \colorrgb{0.30,0.30,0.30}%
      \put(3506,3989){\makebox(0,0){\strut{}Ajuste Mergesort}}%
      \csname LTb\endcsname%
      \put(4376,3486){\makebox(0,0)[r]{\strut{}-1.00e+00$ \cdot x^{8.06e-09}+x \cdot nan$}}%
      \csname LTb\endcsname%
      \put(4376,3266){\makebox(0,0)[r]{\strut{}Mergesort}}%
    }%
    \gplbacktext
    \put(0,0){\includegraphics{./graficos/ajuste-mergesort}}%
    \gplfronttext
  \end{picture}%
\endgroup

	\end{center}
\end{frame}

\begin{frame}
	\fontsize{8pt}{7.2}\selectfont
	\begin{center}
		\input{graficos/ajuste-heapsort}
	\end{center}
\end{frame}

\begin{frame}
	\fontsize{8pt}{7.2}\selectfont
	\begin{center}
		\input{graficos/ajuste-quicksort}
	\end{center}
\end{frame}

\begin{frame}
	\fontsize{7pt}{7.2}\selectfont
	\begin{center}
		\input{graficos/ajuste-floyd}
	\end{center}
\end{frame}

\begin{frame}
	\fontsize{8pt}{7.2}\selectfont
	\begin{center}
		\input{graficos/ajuste-hanoi}
	\end{center}
\end{frame}



\end{document}
