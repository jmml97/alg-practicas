%%
% Please see https://bitbucket.org/rivanvx/beamer/wiki/Home for obtaining beamer.
%%
\documentclass[spanish]{beamer}

%%% CODIFICACIÓN

\usepackage[utf8]{inputenc}
\usepackage[spanish]{babel}



%%% FUENTES

\usepackage[T1]{fontenc}
\usepackage[familydefault,regular]{Chivo}
\usepackage{newtxsf} % Fuente de matemáticas

\setbeamertemplate{navigation symbols}{}

%%% COLORES

\definecolor{background}{RGB}{237,237,237}
\definecolor{text}{RGB}{78,78,78}
\definecolor{accent}{RGB}{129, 26, 24}

% ¿Negrita en el título de diapositiva o no?
%\setbeamertemplate{frametitle}{\color{accent}\vspace*{1cm}\bfseries\insertframetitle\par\vskip-6pt}

\setbeamertemplate{frametitle}{\color{accent}\vspace*{1cm}\insertframetitle\par\vskip-6pt}

\setbeamertemplate{itemize items}[circle]

%%% INFORMACIÓN DEL DOCUMENTO

\title{Algorítmica: práctica 1}
\subtitle{Análisis de la eficiencia de algoritmos}
\author{Sofía Almeida Bruno \and Antonio Coín Castro \and María Victoria Granados Pozo \and Miguel Lentisco Ballesteros \and José María Martín Luque}
\begin{document}

%%% CONFIGURACIÓN DE COLORES DE BEAMER

\setbeamercolor{background canvas}{bg=background}
\setbeamercolor{normal text}{fg=text}
\setbeamercolor{alerted text}{fg=accent}
\setbeamercolor{block title}{fg=accent}
\setbeamercolor{alerted text}{fg=accent}
\setbeamercolor{itemize item}{fg=accent}
\setbeamercolor{enumerate item}{fg=accent}
\setbeamercolor*{title}{fg=accent}
\setbeamercolor{qed symbol}{fg=accent}
\usebeamercolor[fg]{normal text}

\maketitle

\begin{frame}{Test}
	Hola a todos
	
	Me gustan los ponies
	
	Helicóptero
	
	$$ f_x = x^4+5$$
\end{frame}

\begin{frame}
	\begin{itemize}
		\item Unicornio
		\item Pony
		\item Caballo
	\end{itemize}
\end{frame}

\begin{frame}
	\begin{enumerate}
		\item Unicornio
		\item Pony
		\item Caballo
	\end{enumerate}
\end{frame}

\begin{frame}
	\begin{theorem}
		Esto es un teorema.
	\end{theorem}
	\begin{corollary}
		Esto es un corolario.
	\end{corollary}
	\begin{proof}
		$$d((t, x), (t_0, x_0)) = \sqrt{(t-t_0)^2 + (x-x_0)^2} < \varepsilon_0$$
	\end{proof}
\end{frame}

\begin{frame}
	\begin{center}
		\include{graficos/burbuja-linux-O1}
	\end{center}
	
\end{frame}

\begin{frame}
	\begin{center}
		\include{graficos/burbuja-linux-O0}
	\end{center}
	
\end{frame}


\end{document}
